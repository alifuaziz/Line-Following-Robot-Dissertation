\section*{Abstract}

\paragraph{Background}
The investigation of spatial memory is important in the understanding of effect spatial memory pathologies such as Alzheimer's Disease. The underlying mechanisms of spatial memory are investigated with animal models' ability to navigate in maze environments. 
There have been previous maze designs that have allowed the animal model to make a limited number of choices deciding where to navigate to. The Honeycomb Maze is able to make a significant improvement on this by allowing many more decisions to be made by the animal model as well as a greater number of parameters being measured in a single run 
\paragraph{Aim of this work}
The aim of this dissertation is to develop a version of the Honeycomb Maze which is formed of three platforms moving around a hexagonal grid of possible moves for the animal model to make. This was programmed in Python successfully for integration with other libraries also written in Python.
\paragraph{Solution}
A solution was programmed using object oriented paradigm programming. By investigating the geometry of the problem presented, a general path-finding solution was made for navigating through the maze, without collisions between robots taking place. This was integrated into a network of possible moves the robots can access without collision. From this the shortest path for each path-finding target is generated.


\paragraph{Conclusion}
A path-finding solution, for the specific task presented, has been implemented for the \textit{Khepera IV} robots in Python. This allows its use as a tool for investigating animal models of spatial cognition. However, there are limitations in its efficiency which can be addressed upon further work.


% and here is why you should care

